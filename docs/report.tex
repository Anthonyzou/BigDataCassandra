\documentclass[letterpaper]{article}
\usepackage[english]{babel}
\usepackage[utf8]{inputenc}
\usepackage{amsmath}
\usepackage{listings}
\usepackage{graphicx}
\usepackage{hyperref}

\title{Big Data Project Report}
\author{Shenwei Liao, Anthony Ou, Jacqueline Terlaan\\
	CMPUT 391\\
Winter 2014}

\begin{document}
\maketitle
\section{Goals and Objectives}
\begin{itemize}
	\item To successfully implement a distributed database system which
		will quickly and reliably store several terabytes of random
		data into large tables with many columns (attributes).
	\item To distribute the data in the above tables across 8 "nodes"
		(servers), which form our "cluster".
	\item To generate additional tables distributed throughout the cluster
		to be optimally configured for the queries to meet their own
		requirements (see below).

	\item To query the above tables such that all of the following requirements
		are met by five queries.
		\begin{enumerate}
			\item Four of five queries must retrieve data across from all of the distributed nodes.
			\item At least one of the queries must contain at least ten atomic
				conditions (formulas) in the WHERE clause.
			\item At least two of the queries must utilize both the GROUP BY and
				ORDER BY clauses. Since Cassandra does not yet support GROUP BY
				aggregation, the same functionality must be achieved in another
				way.
			\item At least three of the five queries must be range queries which
				specify an upper and lower bound on some ordered attribute that
				is used for the purposes of querying.
			\item None of the five queries can be trivial. That is, there can be
				no simple key searches or anything that is of little practical
				interest for measuring how a database system can deal with
				computationally expensive requests.
		\end{enumerate}

	\item To record the execution time of each query.
\end{itemize}

\section{Methodology, Tools and Equipment Used}
8 server instances were used, each with 32GB virtual memory and 1TB secondary storage 

\section{Implementation}

\section{Shell Script}
A shell script was written to wrap all functionality of the project,
including server setup for all nodes, table creation and population and
executing the queries. To run it, cd into the directory containing group3.sh,
and call "sh group3.sh", and follow the prompts interactively.

\subsection{Configuration Changes}
Many different aspects of Cassandra cluster and the table were tuned so that we
can achieve a high write availability and high performance for generating
a randomly generated dataset. Here are some of the more notable configuration
changes that we applied:
\begin{itemize}
	\item Cluster-wide changes (in cassandra.yaml):
		\begin{enumerate}
			\item Use of RoundRobinScheduler over NoScheduler.
				Having a request scheduler allowed for requests
				to be assigned to nodes faster by the
				coordinating node, speeding up
				concurrent reads and writes.
			\item Reducing commit log sync frequency. Writing to
				the commit log must happen for every
				transaction, as is necessary for write
				durability. For the purposes of this project,
				we can sacrifice some durability to go faster.
			\item Concurrent writes increased.
			\item Memtable flush writers increased. Memtable is the
				in-memory data structure cassandra uses to hold
				batched writes.
			\item Disabling automatic backups and snapshots of the
				keyspace. 
			\item Removed the throttling on compaction throughput.

		\end{enumerate}
	\item Table-specific changes:
		\begin{enumerate}
			\item Replication factor set to 3.
			\item Durable writes are disabled. The commit log
				doesn't need to be written to just for table
				generation. This and the next setting are only
				off because we are randomly generating data and
				there isn't such a thing as a wrong random
				value.
			\item Consistency level for inserts was set to 0 for
				maximum write availability.
			\item Compression is disabled to keep CPU utilization low.
			\item Removing column keys that had low cardinality.
			\item Making indexes after the bulk insert.
		\end{enumerate}
\end{itemize}

\subsection{Table Schemas}
The cdr table follows the assignment specification's schema, with 470 columns.
It 
and 16 column keys, indexed on MONTH\_DAY and MOBILE\_ID\_TYPE.
The role of the last two tables is to make the substitute
aggregate queries both possible.
These tables are not generated in a sparse manner, every column will have an entry of the appropriate type.
\begin{enumerate}
	\item "CDR"
		The large 470 column family with 16
		column keys and a UUID as the partition key. This helps with distributing across nodes
		when the partitioner is random or byte ordered.

	\item "CDR\_ALT"
		Contains the same 470 columns as cdr, but the last 6 column
		keys are no longer column keys. These 6 columns were never queried upon but in the CDR table
		for testing.
		
	\item "SMALLCDR"
		This table has the last 235 columns removed but has the same 10 column keys as CDR\_ALT,
		an experiment in partitioned table.

	\item "GROUP\_BY\_MONTH" 
		\begin{itemize}
			\item Keys are between 1 and 31.
			\item UUID of the partition key in CDR is kept as a column key
			\item Reduces data to facilitate completion of
				a query equivalent to one in SQL
				containing `GROUP BY MONTH\_DAY'.
			\item Ordering is implicitly done by Cassandra
		\end{itemize}
	\item "GROUP\_BY\_MOBILE\_ID\_TYPE"
		\begin{itemize}
			\item Keys are between 0 and 7.
			\item Again the UUID of the partition key in CDR is kept as a column key
			\item Similar to "GROUP\_BY\_MONTH", but instead
				the assignment
				is based on its insertion number modulo
				8. This helps us identify if insertions are consistent since 
				this number should not variate much.
				
			\item Ordering is implicitly done by Cassandra
		\end{itemize}
		
	\item Indexes
		 An alternative to creating a table for grouping by.
		\begin{itemize}
			\item On table CDR, column month\_day
			\item On table CDR, column mobile\_id\_type
		\end{itemize}
		
\end{enumerate}

\subsection{Programs}

We wrote two main Python 2.7 scripts: generate.py and query.py. These programs are meant to run in
the background uninterrupted.
The generate program is designed such that it can run many times as a different process with a
different seed, so that data generation doesn't collide.
A benefit is that now a generator process can be run on each node to maintain a balanced CPU load.
Query.py runs the queries with facilities to measure real time spent running.

\subsection{CQL Queries}

\begin{enumerate}
	\item Query 1
		\lstinputlisting[breaklines=true,frame=single,language=SQL]{queries/query1.sql}

		Cluster-wide range query with 10 atomic formulas. Searches on
		every column key of the CDR table, as this is the only
		conceivable way to properly execute a range query.  Returns
		results from each node, satisfies the range query requirement
		and has 10 atomic formulas.

		Potential uses: many 'outliers' are eliminated from the
		selection because they are above and below the minimum and
		maximum values of all the columns
		respectively, by approximately 10\%. In other words, any
		'extreme' data points in the CDR table can be ignored for
		statistical analysis.

		Query 1 ('Alternate')

		Performs the same cluster-wide query as Query 1, but over the table cdr\_alt.

		Query 1 ('Small')

		This query
		is intended to take less time than the previous query, because fewer columns
		have to be read into memory in order for the atomic conditions to be evaluated.

	\item Query 2
		\lstinputlisting[breaklines=true,frame=single,language=SQL]{queries/query2.sql}

		Satisfies two requirements: cluster-wide and range query. Retrieves the number of
		cdr entries for cities with IDs strictly greater than 5000 and strictly less than
		90000. 

		Potential uses: In some cases companies may wish to know how many calls were made
		in a given range of cities, based on certain CITY\_IDs. While
		those cities may not have any major features in common, this
		query can be used in cases where only a small
		sample of all cities is required.

	\item Query 3
		\lstinputlisting[breaklines=true,frame=single,language=SQL]{queries/query3.sql}

		Satsifies two requirements: cluster-wide and range query. Counts the number of cdr
		entries such that DUP\_SEQ\_NUM is strictly greater than 30000 and strictly less than
		300000. All other atomic conditions in the range query are used to ensure that
		all of the results fall within a valid range and have no null-valued entries for
		CITY\_ID, SERVICE\_NODE\_ID, and so on.

		Potential uses: Suppose that statistical data is to be generated for rows in the
		CDR table for which there are no null-valued entries in the columns CITY\_ID,
		SERVICE\_NODE\_ID, RUM\_DATA\_NUM, MONTH\_DAY and DUP\_SEQ\_NUM, and that DUP\_SEQ\_NUM
		falls within a more specific range, that is 30000-300000.
		\\\\\\\\
	\item Query 4
		\lstinputlisting[breaklines=true,frame=single,language=python]{queries/query4.sql}

		Satisfies two requirements: group-by and order-by clause, and cluster-wide query.

		A table of UUIDs aggregates the keys of the MOBILE\_ID\_TYPE columns.
		This query has to be in a python loop to execute a query on each group.
		

		Additionally this is a idempotent transaction, this insert can be done multiple times and it 	
		will not cause inconsistencies.

		Inserting into the table is done in the following way:
		\begin{lstlisting}[frame=single]
insert into group_by_MOBILE_ID_TYPE 
(MOBILE_ID_TYPE, id) values (?,?)
		\end{lstlisting}
		
		Potential uses: Most likely for the generation of histograms, but other basic
		kinds of statistical analysis can be performed on these small MOBILE\_ID\_TYPE 
		'buckets' as well, such as calculating arithmetic averages, etc.

	\item Query 5
		\lstinputlisting[breaklines=true,frame=single,language=python]{queries/query5.sql}

		Satisfies the group-by and order-by clause requirements. 
		'CLUSTERING ORDER BY (MONTH\_DAY)' ensures that the GROUP\_BY\_MONTHs entries are ordered
		by MONTH\_DAY, and GROUP BY functionality is achieved via aggregating UUIDs of each 
		row in CDR.
		
		Potential uses: finding out which day(s) of each month of the year have the highest
		frequency of entries added to the CDR table can be of use for marketing to customers,
		optimizing billing rates, and so on. Similar queries can be written for different
		times of the day, months or weeks of the year, by aggregating CDR row counts into 
		hourly, monthly or weekly groupings for more specific results which could be of use 
		in billing plans with calls at different times incurring different costs.
\end{enumerate}

\section{Experimental Results}

\subsection{Table Population}

The biggest table to generate took two and a half days, generating an average
of 380GB per node for a total of just over 3TB of data, out of the 8TB that was
available on the cluster, with roughly 8.8 million columns. 
The average speed of insertions to the big table was roughly 15MB/s.

A separate keyspace was used for some postliminary tests. With the additional smallcdr table
to mimic a sparse table or a partitioned table and cdr\_alt table changed to sort ascendingly. 
Insertion of about 700000 entries --- resulted in about 40GB of data per node which took 3 hours.

\subsection{Query Results}
When it comes to searching very large data sets like the 400GB node with a potentially inconsistent read. 
The quickest reads came from maximizing column keys usage
like in query 1 or minimizing column key usage like in query 2. Going half way into the column keys, 
happened to be the slowest possibly query for the 400GB dataset, while in the other set
it became the quickest query overall. 

Doing consistent reads follows the same behavior except the speed of 
queries that minimize column key usage is lost, i.e query 2.
For the 40GB data set the results were very similar except we see the effects of reducing column keys and
columns in the smallcdr table, this table led to the fastest query one.
An anomaly is how query 3 was the slowest query in the 400GB data set, but now it is actually the 
fastest query in this dataset.

The GROUP BY queries predictably executed the
fastest, as the work was already done at insertion time. 
On lower consistency levels queries on this table is always reduces query time.
\\

\begin{tabular}[h]{|l|r|r|}
\hline
& 400GB CsistLvl=3 & 400GB CsistLvl=1 \\ \hline
Query 1       & 3.032994934 & 2.200000453 \\
Q. 1 Alt,desc   & 6.137618132 & 4.19432343245 \\
Query 2       & 4.171009902 & 2.12728600105 \\
Query 3       & 9.964161499 & 4.75602963368 \\
Query 4       & 0.9116 & 0.6132 \\
Query 5       & 0.8965 & 0.5538 \\\hline
\end{tabular}

Fig 1. Queries on initial keyspace with a 400GB load per node and a replication factor 3 with units in minutes. Queries 1-3 return 4000 results.
\\

\begin{tabular}[h]{|l|r|r|}
\hline
& 40GB CsistLvl=3 & 40GB CsistLvl=1 \\ \hline
Query 1       & 3.76711789767 & 1.394323281 \\
Q. 1 Alt,asc    & 5.64526818196 & 4.377174052 \\
Query 1 Small  & 2.46129718224 & 2.025322433 \\
Query 2        & 4.20441953341 & 1.726343501 \\
Query 3        & 1.50994411707 & 0.5942962646 \\
Query 4       & 0.1503 & 0.1344 \\
Query 5       & 0.1674 & 0.1395 \\
\hline
\end{tabular}

Fig 2. Second keyspace made with cdr\_alt sorted ascendingly and the smaller cdr table.

\section{Discussion}

What is not obvious from query one and derivatives, is as columns count increases, so should the number 
of column keys to keep query time from growing exponentially.
We saw the move from 470 columns with 16 column keys in the cdr table to 470 columns with 10 column
keys in cdr\_alt, and this resulted in a
longer query time by over a factor of two, when reducing the amount of column keys by over 50\%.

In addition we saw the move from the cdr\_alt table with 470 columns and 10 column keys to the smallcdr table with 235 columns and 10 column keys which
reduced the query time by over 50\%. 

With this information in mind, if one wanted to denormalize their data and the data characteristics was
not very sparse, it could be a valid strategy to put in dummy columns keys that is never sparse at the
end of the column key (at the end because matters when declaring column keys). In addition we also
saw how querying up until the middle of the column keys was incredibly quick or incredibly slow, if 
this could be consistently leveraged it would could half read time.

Searching on indexes made on the CDR column to facilitate a group by query could have been done, but
it was obvious such a procedure would be incredibly slow. A procedure like this would be viable for 
redoing the entire schema of Cassandra.

\section{Conclusions}

The bulk insertion phase of the test had to be tuned in server side and in schema related ways, to maximize
insertions.
The most important server side changes that affected insert speed was using the round
robin scheduler to have more nodes handle requests in parallel, to disable
the commit log/durable writes in order to get more disk IO, and disabling compression of tables helped
reduce CPU utilization allowing for more write threads. Schema configurations that were found to be
bottlenecks were putting low carnality columns as column keys and making indexes after the bulk 
insertion.

In general denormalizing the data is not viable without increasing column keys or ensuring the data
was sparse. It might be reasonable to assume that wide sparse tables inherently achieves normalized
schema characteristics since nulls take up no space in Cassandra, and this works well in a column 
oriented database. 
A guess as to why increasing column keys helps speed up queries, is that columns keys can help buffer
earlier column keys and prevent Cassandra from reading a mix of column keys and regular columns.
Of course that doesn't mean all column keys are good, it is possible a column key is polluting
another column key because it is unrelated to the query. This explains how going into the middle of
the column key list can be very slow like in query 3 depending on the data, or very fast if the data
is aligned correctly. There is probably the biggest difference, when querying in the middle of the
column families, because these columns are never empty. 

\section{Notes on Possible Improvements}

Cassandra offers the ability of store entire tables into memory, using this feature on the
group by tables would have likely led to a large speed up, and would have been
interesting to explore, but a pure implementation of this would probably not be big data.

It is possible to run the server with a custom scheduling, so with this in mind it could be possible
to write a scheduler that is optimized for bulk insertions.
Cassandra also offers trigger support which could automatically reduce inserts into
appropriate tables. These previous two features aren't well documented but could be useful in future
works.

\section{References Used}

Agilent Technologies Inc (28 October, 2014), "Mobile Station Reported Pilot Information".
\url{http://wireless.agilent.com/rfcomms/refdocs/cdma2k/c2kla_gen_ms_pilot_meas_report.html}
Accessed: 10 March 2014.

Apache Software Foundation (2004), "The Apache Licence, Version 2.0".
\url{http://www.apache.org/licences/LICENCE-2.0}
Accessed 7 March, 2014.

Apache Software Foundation (10 March 2014), "Cassandra Query Language (CQL) v3.1.5".
\url{http://cassandra.apache.org/doc/cql3/CQL.html}
Accessed: 10 March 2014.

Apache Software Foundation (15 November 2013), "UUID".
\url{http://wiki.apache.org/cassandra/UUID}
Accessed: 13 March 2014.

DataStax Documentation (2014), "Cassandra Storage Basics".
\url{http://www.datastax.com/documentation/cassandra/2.0/cassandra/dml/manage_dml_intro_c.html}
Accessed 7 March, 2014.

DataStax Documentation (2014), "The cassandra.yaml Configuration File".
\url{http://www.datastax.com/documentation/cassandra/2.0/cassandra/configuration/configCassandra_yaml_r.html}
Accessed 7 March, 2014.  

DataStax Documentation (2014), "CLI keyspace and table storage configuration".
\url{http://www.datastax.com/documentation/cassandra/2.0/cassandra/reference/referenceStorage_r.html}
Accessed: 10 March 2014.

DataStax Inc. (2014), "UUID and timeuuid".
\url{http://www.datastax.com/documentation/cql/3.0/cql/cql_reference/uuid_type_r.html}
Accessed: 13 March 2014.

DataStax Documentation (2014), "What's new in Cassandra 2.0".
\url{http://www.datastax.com/documentation/cassandra/2.0/cassandra/features/features_key_c.html}
Accessed: 10 March 2014.

DataStax Documentation (2014), "The Write Path to Compaction".
\url{http://www.datastax.com/documentation/cassandra/2.0/cassandra/dml/dml_write_path_c.html}
Accessed 7 March, 2014.

Fowler, M. (9 January 2012), "NosqlDefinition".
\url{http://martinfowler.com/bliki/NosqlDefinition.html}
Accessed: 13 March 2014.

Fowler, M. (16 November 2011), "PolyglotPersistence".
\url{http://martinfowler.com/bliki/PolyglotPersistence.html}
Accessed: 13 March 2014.

Fowler, M. (7 January 2013), "Schemaless Data Structures".
\url{http://martinfowler.com/tags/noSQL.html}
Accessed: 14 March 2014.

Grinev, M. (9 July 2010), "A Quick Introduction to the Cassandra Data Model".
\url{http://maxgrinev.com/2010/07/09/a-quick-introduction-to-the-cassandra-data-model/}
Accessed: 10 March 2014.

Grinev, M. (12 July 2010), "Do You Really Need SQL to Do It All in Cassandra?".
\url{ http://maxgrinev.com/2010/07/12/do-you-really-need-sql-to-do-it-all-in-cassandra/}
Accessed: 10 March 2014.

McFadin, P. (13 November 2012), "Cassandra Data Modeling Talk".
\url{http://www.slideshare.net/patrickmcfadin/data-modeling-talk}
Accessed: 13 March 2014.

McFadin, P. (3 May 2013), "The Data Model is Dead, Long Live the Data Model!".
\url{http://www.slideshare.net/patrickmcfadin/the-data-model-is-dead-long-live-the-data-model}
Accessed: 14 March 2014.

McFadin, P. (16 May 2013), "Become a Super Modeler".
\url{http://www.slideshare.net/patrickmcfadin/become-a-super-modeler}
Accessed: 14 March 2014.

McFadin, P. (24 June 2013), "The World's Next Top Data Model".
\url{http://www.slideshare.net/patrickmcfadin/the-worlds-next-top-data-model}
Accessed: 14 March 2014.

McFadin, P. (11 October 2013), "Cassandra 2.0: Better, Stronger Faster".
\url{http://blog.imaginea.com/consistency-tuning-in-cassandra/}
Accessed: 14 March 2014.

McFadin, P. (17 February 2014), "Time Series with Apache Cassandra".
\url{http://www.slideshare.net/patrickmcfadin/time-series-with-apache-cassandra-strata}
Accessed: 13 March 2014.

Oracle Inc. (Copyright 1996-2005), "Datatype Comparison Rules".
\url{http://docs.oracle.com/cd/B19306_01/server.102/b14200/sql_elements002.htm#i55214}
Accessed: 10 March 2014.

Oracle Inc (No date), "GROUP BY clause".
\url{http://docs.oracle.com/javadb/10.6.1.0/ref/rrefsqlj32654.html}
Accessed 7 March, 2014.

Oracle Inc (No date), "ORDER BY clause".
\url{http://docs.oracle.com/javadb/10.6.1.0/ref/rrefsqlj13658.html}
Accessed: 10 March 2014.

Oracle Inc (No date), "SelectExpression".
\url{http://docs.oracle.com/javadb/10.6.1.0/ref/rrefselectexpression.html#rrefselectexpression}
Accessed: 10 March 2014.

Oracle Inc. (Copyright 1996-2005), "SYSDATE".
\url{http://docs.oracle.com/cd/B19306_01/server.102/b14200/functions172.htm}
Accessed: 10 March 2014.

Oracle Inc. (Copyright 1996-2005), "SYSTIMESTAMP".
\url{http://docs.oracle.com/cd/B19306_01/server.102/b14200/functions173.htm}
Accessed: 10 March 2014.

Oracle Inc. (Copyright 1996-2005), "TO\_DATE".
\url{http://docs.oracle.com/cd/B19306_01/server.102/b14200/functions183.htm}
Accessed: 10 March 2014.

Saxena, S. (4 September 2013), "Consistency Tuning In Cassandra".
\url{http://blog.imaginea.com/consistency-tuning-in-cassandra/}
Accessed: 14 March 2014.


\end{document}
